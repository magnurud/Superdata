%%%%%%%%%%%%%%%%%%%%%%%%%%%%%%%%%%%%%%%%%
% Short Sectioned Assignment
% LaTeX Template
% Version 1.0 (5/5/12)
%
% This template has been downloaded from:
% http://www.LaTeXTemplates.com
%
% Original author:
% Frits Wenneker (http://www.howtotex.com)
%
% License:
% CC BY-NC-SA 3.0 (http://creativecommons.org/licenses/by-nc-sa/3.0/)
%
%%%%%%%%%%%%%%%%%%%%%%%%%%%%%%%%%%%%%%%%%

%----------------------------------------------------------------------------------------
%	PACKAGES AND OTHER DOCUMENT CONFIGURATIONS
%----------------------------------------------------------------------------------------

\documentclass[norsk]{article} % A4 paper and 11pt font size

\usepackage[T1]{fontenc} % Use 8-bit encoding that has 256 glyphs
\usepackage{fourier} % Use the Adobe Utopia font for the document - comment this line to return to the LaTeX default
\usepackage[english]{babel} % English language/hyphenation
\usepackage{amsmath,amsfonts,amsthm} % Math packages

% Added by Haavard %---------------------------------------------------------------------
\usepackage[utf8]{inputenc} % Norwegian letters
\usepackage{fullpage}
\usepackage{subcaption}
\usepackage[font={small, it}]{caption} % captions on figures and tables
\usepackage{graphicx}
\usepackage{color}
\usepackage{hyperref} % Use \autoref{ and \nameref{
\hypersetup{backref,
  colorlinks=true,
  breaklinks=true,
  %hidelinks, %uncomment to make links black
  linkcolor=blue,
  urlcolor=blue,
  citecolor=blue
}
\usepackage[all]{hypcap} % Makes hyperref jup to top of pictures and tables
% --------------------------------------------------------------------------------------

\usepackage{lipsum} % Used for inserting dummy 'Lorem ipsum' text into the template

\usepackage{sectsty} % Allows customizing section commands
\allsectionsfont{\centering \normalfont\scshape} % Make all sections centered, the default font and small caps

\usepackage{fancyhdr} % Custom headers and footers
\pagestyle{fancyplain} % Makes all pages in the document conform to the custom headers and footers
\fancyhead{} % No page header - if you want one, create it in the same way as the footers below
\fancyfoot[L]{} % Empty left footer
\fancyfoot[C]{} % Empty center footer
\fancyfoot[R]{\thepage} % Page numbering for right footer
\renewcommand{\headrulewidth}{0pt} % Remove header underlines
\renewcommand{\footrulewidth}{0pt} % Remove footer underlines
\setlength{\headheight}{13.6pt} % Customize the height of the header

\numberwithin{equation}{section} % Number equations within sections (i.e. 1.1, 1.2, 2.1, 2.2 instead of 1, 2, 3, 4)
\numberwithin{figure}{section} % Number figures within sections (i.e. 1.1, 1.2, 2.1, 2.2 instead of 1, 2, 3, 4)
\numberwithin{table}{section} % Number tables within sections (i.e. 1.1, 1.2, 2.1, 2.2 instead of 1, 2, 3, 4)

\setlength\parindent{0pt} % Removes all indentation from paragraphs - comment this line for an assignment with lots of text

%----------------------------------------------------------------------------------------
%	TITLE SECTION
%----------------------------------------------------------------------------------------

\newcommand{\horrule}[1]{\rule{\linewidth}{#1}} % Create horizontal rule command with 1 argument of height

\title{	
\normalfont \normalsize 
\textsc{NTNU} \\ [25pt] % Your university, school and/or department name(s)
\horrule{0.5pt} \\[0.4cm] % Thin top horizontal rule
\huge TMA4280 - project part 2\\ % The assignment title
\horrule{2pt} \\[0.5cm] % Thick bottom horizontal rule
}

\author{Håvard Kvamme, Jørgen Vågan, Magnus Aarskaug Rud} % Your name

\date{\normalsize\today} % Today's date or a custom date

\begin{document}

\maketitle % Print the title

%----------------------------------------------------------------------------------------
%	PROBLEM 1
%----------------------------------------------------------------------------------------
\abstract{In this project the Kongull cluster have been used to solve the homogenous 
					2D Poisson problem on the unit square, both Message sending and shared memory 
				  solution models have been implemented and compared.}

\section{Problem Description}

In this project the task is to solve the two-dimensional poisson problem as stated below.
\begin{align}
	-\nabla^2 u &= f  \text{ in } \Omega = (0,1)\times(0,1) \\
	u &=  0 \text{ on } \partial \Omega
\end{align}
%
The problem is solved using a fast diagonalization method. The derivation of the method can be found in \cite{poisson} ,
and the system of equation to be solved can be written as 
\begin{align}
	\underline{T} \; \underline{U} + 	\underline{U} \; \underline{T} = \underline{G} 	
	\label{eq:Matrix}
\end{align}
where $\underline{T}$ is the discrete second order partial differential operator in one direction, 
$\underline{G}$ is the discrete loading function multiplied with the steplength squared $h^2$ and $\underline{U}$ is the discrete solution
to the problem.
and the method can be summed up into the following three steps:


\begin{description}
	\item[step 1  -  $O(n^2\log(n))$] \hfill \\ \hfill \\
		Compute $\underline{\tilde{G}} = \underline{Q}^T\underline{G} \; \underline{Q} $    -    two matrix-matrix products 
		\hfill \\
	\item[step 2  -  $O(n^2)$] \hfill \\ \hfill \\
		Compute $\tilde{u}_{i,j} = \frac{\tilde{g}_{i,j}}{\lambda_i+\lambda_j}$    -    scalar addition and divition
		\hfill \\
	\item[step 3  -  $O(n^2log(n))$] \hfill \\ \hfill \\
		Compute $\underline{U} = \underline{Q}\underline{\tilde{U}} \; \underline{Q}^T $    -     two matrix-matrix products 
\end{description}

$\underline{Q}$ and $\underline{\Lambda}$ contains the eigenvectors and the eigenvalues of the discrete 
second order partial differential operator. 

Step 1 and 3 are solved using the infamous fast sine transform.

\section{Implementation strategy}
%
The implementation was done in C and can be found in the file \verb+poisson-mpi.c+. The whole algorithm focuses on the tree steps in the previous section. As it should be able to run on supercomputers, it was necessary to be able to use distributed memory. This was solved using MPI. To be able to do \textbf{step 1} and \textbf{step 3} with the fast sine transform, the external code \verb+fst.f+ was used. However, this code demands a problem size (number of rows) $n = 2^k$, where $k$ is some integer. It also needed to work on a whole column at a time. The way \textbf{step 1} (\textbf{step 3} is solved exactly the same way) is done with the fast sine transform is as follows
%
\begin{align}
\underline{\tilde{G}} = \underline{S}^{-1} \left( (\underline{S} \: \underline{G})^T   \right),
\end{align}
%
where $S$ and $S^{-1}$ is the fast- and inverse fast sine transform respectively. So this is done in three steps
%
\begin{align}
  \underline{G} &\leftarrow \underline{S} \: \underline{G},\\
  \label{eq:transp} 
  \underline{A} &\leftarrow \underline{G}^T,\\
  \underline{\tilde G} &\leftarrow \underline{S}^{-1} \: \underline{A}.
\end{align}
%
The second step \eqref{eq:transp} will require sending between processes as long as a column, row, or block layout is used. Therefore it is reasonable to distribute the matrices column-wise among the processes. As $n$ is a power of two, the number of internal nodes, and therefore the size of the matrices, are $m = n-1$ which is odd. When $m$ is not a multiple of the number of processes $p$, the extra columns are given to the last process. One could have distributed them more evenly among the processes, but this was tested and did not have a significant impact on the computation time as $n >> p$.

As the fast sine transform need columns, the matrices was allocated column-wise in an attempt to make good use of cache. When transposing the $\underline{G}$, it would better with a row mayor layout, but as there is only one transpose operation, but a total to 2 fast sine transforms and 2 inverse fast sine transforms, the column-mayor layout seemed the best choice. \\
\\
When transposing the matrix, all the processes need to send data to each other. This is in principle done as in Figure~\ref{fig:mpisend}. However, as the matrix of each process is structured column-wise, the local matrix ($m \times r_i$ for process $i$) is first transposed locally (to $r_i \times m$).
%
\begin{figure}[h!]
\begin{center}
    \includegraphics[angle=90,scale=0.35]{./Figures/matrix_blocktranspose.pdf}
\end{center}
\caption{The transpose operation using message passing: the packing and unpacking of data. The figure is due to Bjarte Hægland.}
\label{fig:mpisend}
\end{figure}




\colorbox{yellow}{Need to address why openMP is slower than MPI}

\section{Description of kongull cluster}
The Kongull cluster is a CentOS 5.3 linux cluster. It has 113 physical nodes, 
where 1 is used for login and 4 is used for I/O. The remaining 108 nodes are 
used for computing. The nodes are equipped with 2$\times$ 6-core 2.4 GHz processors, with 6$\times$ 512KiB 
(1 kibibyte = 1024 byte) L1-cache and a common 6 MiB (1 mebibyte = $1024^2$ bytes) L3-cache.
The compute nodes are spread out on 4 racks, 
(rack 0: 40 nodes, rack 2: 40 nodes, rack 1: 16 nodes; rack 3: 12 nodes) 
where the first 3 hold the HP AMP Opteron based nodes (2.4 GHz core speed) 
%\todo{clock speed?}
, while the last rack 
(rack 3) holds the new Dell intel based nodes (2$\times$ 2.60 GHz 8-core processors). In total 1344 compute nodes, with a ''Fat Tree'' network layout.
%\todo{is this okay? do i need a reference? https://www.hpc.ntnu.no/display/hpc/Kongull+Hardware }

% FAT TREE REF:  http://clusterdesign.org/fat-trees/
% KONGULL REF: http://clusterdesign.org/fat-trees/
%full name
%system
%type
%number of nodes
%a single node: details etc

%number of cores physical
%number of cores logical
%(logical cores = (physical cores)x(number of threads that can run through hyperthreading)
%(hyperthreading: For each physical core the OS addresses two virtual/logical cores and shares
%the workload between them when possible. The main function of hyperthreading is to increase 
%the number of independent instruction in the pipeline; it takes advantage of superscalar 
%architecture(inctruction level parallelism) in which multiple instructions operate
%on separate data in parallel.)
%CPU type
%theoretical total peak
%weight.

\section{Results}
Presentation of graphs, vary p from 1 to 36
and $n= 2^{14}$

\begin{figure}[h!]
  \centering
  \begin{subfigure}[b]{0.48\textwidth}
    \includegraphics[width=\textwidth]{./Figures/taskc1.pdf}
  \end{subfigure}%
  \quad
  \begin{subfigure}[b]{0.48\textwidth}
    \includegraphics[width=\textwidth]{./Figures/taskc2.pdf}
  \end{subfigure}
          %(or a blank line to force the subfigure onto a new line)
  \vspace{1\baselineskip}
  \caption{Times (in seconds) for running MPI processes vs threads with $n = 2^{14}$. The total number of processors used are 12 per node, so the number of processes per node is 12 - threads. The left figure is run on one node, while the right is run on three nodes. The red crosses are run without any MPI sending at all.}
  \label{fig:taskc}
\end{figure}
%
\begin{figure}[h!]
  \centering
  \begin{subfigure}[b]{0.48\textwidth}
    \includegraphics[width=\textwidth]{./Figures/taskbTimeProc1.pdf}
  \end{subfigure}%
  \quad
  \begin{subfigure}[b]{0.48\textwidth}
    \includegraphics[width=\textwidth]{./Figures/taskbTimeProc2.pdf}
  \end{subfigure}
  \quad
  \begin{subfigure}[b]{0.48\textwidth}
    \includegraphics[width=\textwidth]{./Figures/taskbTimeNodesTimesThreads.pdf}
  \end{subfigure}
          %(or a blank line to force the subfigure onto a new line)
  \vspace{1\baselineskip}
  \caption{Times for running problem with different amount of processes. In the upper left figure each process has one thread, while in the upper right, each process has two treads. The problem is run on as few nodes as possible, and the processes are identically distributed among the nodes. The problem size $n$ is specified in the plots. In the bottom figure only one MPI process is run per node. Between 1 and 12 threads are run on each node.}
  \label{fig:time}
\end{figure}
%
\begin{figure}[h!]
  \centering
  \begin{subfigure}[b]{0.48\textwidth}
    \includegraphics[width=\textwidth]{./Figures/taskbSpeedupProc1.pdf}
  \end{subfigure}%
  \quad
  \begin{subfigure}[b]{0.48\textwidth}
    \includegraphics[width=\textwidth]{./Figures/taskbSpeedupProc2.pdf}
  \end{subfigure}
  \quad
  \begin{subfigure}[b]{0.48\textwidth}
    \includegraphics[width=\textwidth]{./Figures/taskbSpeedupNodesTimesThreads.pdf}
  \end{subfigure}
          %(or a blank line to force the subfigure onto a new line)
  \vspace{1\baselineskip}
  \caption{Speedup for running problem with different amount of processes. In the upper left figure each process has one thread, while in the upper right, each process has two treads. The problem is run on as few nodes as possible, and the processes are identically distributed among the nodes. There are drawn vertical lines to show when a new node is utilized, and a line with slope 1. The problem size $n$ is specified in the plots. In the bottom figure only one MPI process is run per node. Between 1 and 12 threads are run on each node.}
  \label{fig:Speedup}
\end{figure}
%
\begin{figure}[h!]
  \centering
  \begin{subfigure}[b]{0.48\textwidth}
    \includegraphics[width=\textwidth]{./Figures/taskbEfficiencyProc1.pdf}
  \end{subfigure}%
  \quad
  \begin{subfigure}[b]{0.48\textwidth}
    \includegraphics[width=\textwidth]{./Figures/taskbEfficiencyProc2.pdf}
  \end{subfigure}
          %(or a blank line to force the subfigure onto a new line)
  \vspace{1\baselineskip}
  \caption{Efficiency for running problem with different amount of processes. In the upper left figure each process has one thread, while in the upper right, each process has two treads. The problem is run on as few nodes as possible, and the processes are identically distributed among the nodes. There are drawn vertical lines to show when a new node is utilized, and a horizontal line at 1. The problem size $n$ is specified in the plots. In the bottom figure only one MPI process is run per node. Between 1 and 12 threads are run on each node.}
  \label{fig:Efficiency}
\end{figure}
%
\begin{figure}[h!]
  \centering
  \begin{subfigure}[b]{0.48\textwidth}
    \includegraphics[width=\textwidth]{./Figures/errVsn.pdf}
  \end{subfigure}%
  \quad
  \begin{subfigure}[b]{0.48\textwidth}
    \includegraphics[width=\textwidth]{./Figures/timeOverN2Vsn.pdf}
  \end{subfigure}
          %(or a blank line to force the subfigure onto a new line)
  \vspace{1\baselineskip}
  \caption{The left figure shows a loglog plot of the error as function of $n$. A reference line with slope $-2$ is drawn. The problem is run on two threads, with the number of processes specified in the legend. Only one node is used. The right figure hows $time/n^2$.}
  \label{fig:conv}
\end{figure}




%\section{Analysis}

speedup calculations, theoretic and observed



\section{Discussion}

\subsection{Bottlenecks}

\subsection{The non-homogenous case}
If one were to solve the Poisson problem with non-homogenous boundary conditions ie. $u = g(x) $ on $\partial \Omega$, 
some minor modifications would have to be done. 
The matrix system that is generated through this algorithm does not include the effect of the boundary elements, 
and in practice this is the same as assuming that they are iqual zero. The second order differential operator on the elements 
closest to the boundary are on the form 
\begin{equation}
	-\frac{\partial^2 u_{i,j}}{\partial x^2} = -\frac{u_{i-1,j}-2u_{i,j}}{h^2}
\end{equation}
The incrementations of $i,j$ will depend on which boundary the element is close to, but the form of the operator is nevertheless the same.
The effect on equation REF MAIN MATRIX EQ is that the boundary element is added to the left side.
In the homogenous case this does not affect the numerical algorithm, but in the non-homogenous case the boundary value needs to be added to the 
right hand side as well. This is implemented before the FST is done and does not have any impact on the parallel implementation. 

\subsection{Variations in the loading function $f$}
The loading function simply needs to be evaluated in all the grid points and multiplied by the steplength, before the 
FST-procedure starts. The only thing that will differ from the serial code is the displacement each process have.

\subsection{Variations in the domain $\Omega$}
By defining the domain $\Omega = (0,L_x)\times(0,L_y)$ but keeping the same number of grid points in each direction 
the discretization of the laplacian operator would be changed. In the unit square the discrete laplacian is defined as 
\begin{equation}
	\Delta u_{i,j} = \frac{u_{i-1,j}-2u_{i,j}+u_{i+1,j}}{h^2}+\frac{u_{i,j-1}-2u_{i,j}+u_{i,j+1}}{h^2}.
\end{equation}
In the new domain the operator would be defined as  
\begin{equation}
	\Delta u_{i,j} = \frac{u_{i-1,j}-2u_{i,j}+u_{i+1,j}}{h_x^2}+\frac{u_{i,j-1}-2u_{i,j}+u_{i,j+1}}{h_y^2}.
\end{equation}
Where $h_x=L_x/n$ and $h_y=L_y/n$ defines the new steplengths in each direction. Notice that the infamous 5-point formula now takes quite a 
different form and some rewriting is necessary. 
The diagonalized matrix system will now end up looking as 
\begin{equation}
	\left( \frac{1}{h_x^2}\underline{T} \; \underline{U}+\frac{1}{h_y^2}\underline{U}\; \underline{T} \right) _{i,j}=f_{i,j}
\end{equation}

By continuing the diagonalization procedure st. $\underline{T}=\underline{Q}\;\underline{\Lambda}\;\underline{Q}^T $
and defining $\underline{\tilde{U}}= \underline{Q}^T\;\underline{U}\;\underline{Q}$ one is left with the matrix system 

\begin{equation}
	\frac{1}{h_x^2}\underline{\Lambda} \; \underline{\tilde{U}}+\frac{1}{h_y^2}\underline{\tilde{U}}\; \underline{\Lambda} =\underline{\tilde{F}}.
\end{equation}

Notice that the right hand side can not be initially multiplied with the steplength $h^2$ since the two terms on the left hand side
now have different coefficients. This has to be done in step 2 of the algorithm, and would be implemented in the following way; 

\begin{equation}
	\tilde{u}_{i,j} = \frac{\tilde{f}_{i,j}}{\lambda_i/h_x+\lambda_j/h_y}.
\end{equation}

The last part of the algorithm would not need any further changing.


\clearpage
\begin{thebibliography}{9}

		\bibitem{poisson}
		Einar M. Rønquist,
		\emph{The Poisson problem in $\mathbb{R}^2$: Diagonalization methods},
		Department of Mathematical Sciences,
		NTNU,N-7491 Trondheim, Norway,
		Revised by Arne Morten Kvarving, 2014.

\end{thebibliography}

\end{document}

