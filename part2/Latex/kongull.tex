\section{Description of kongull cluster}
The Kongull cluster is a CentOS 5.3 linux cluster. It has 113 physical nodes, 
where 1 is used for login and 4 is used for I/O. The remaining 108 nodes are 
used for computing. The nodes are equipped with 2$\times$ 6-core 2.4 GHz processors, with 6$\times$ 512KiB 
(1 kibibyte = 1024 byte) L1-cache and a common 6 MiB (1 mebibyte = $1024^2$ bytes) L3-cache.
The compute nodes are spread out on 4 racks, 
(rack 0: 40 nodes, rack 2: 40 nodes, rack 1: 16 nodes; rack 3: 12 nodes) 
where the first 3 hold the HP AMP Opteron based nodes (2.4 GHz core speed) 
%\todo{clock speed?}
, while the last rack 
(rack 3) holds the new Dell intel based nodes (2$\times$ 2.60 GHz 8-core processors). In total 1344 compute nodes, with a ''Fat Tree'' network layout.
%\todo{is this okay? do i need a reference? https://www.hpc.ntnu.no/display/hpc/Kongull+Hardware }

% FAT TREE REF:  http://clusterdesign.org/fat-trees/
% KONGULL REF: http://clusterdesign.org/fat-trees/
%full name
%system
%type
%number of nodes
%a single node: details etc

%number of cores physical
%number of cores logical
%(logical cores = (physical cores)x(number of threads that can run through hyperthreading)
%(hyperthreading: For each physical core the OS addresses two virtual/logical cores and shares
%the workload between them when possible. The main function of hyperthreading is to increase 
%the number of independent instruction in the pipeline; it takes advantage of superscalar 
%architecture(inctruction level parallelism) in which multiple instructions operate
%on separate data in parallel.)
%CPU type
%theoretical total peak
%weight.
