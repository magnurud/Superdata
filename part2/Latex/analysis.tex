\section{Analysis}

The time it will take to solve the problem with the implemented algorithm is of order $O(n^2\log(n))$.
Notice that as the problem size increases $\log(n)$ will be very small compared to $n^2$. It is therefore expected to see
the time increase almost as a function of $n^2$ for larger problem sizes.

The speedup $S_p$ of a program using $P$ processes is given as $S_p=\frac{T_1}{T_p}$. 
The theoretical upper bound of this value is the number and processes, and will often be much lower 
due to necessary serial code and sending costs. In the code used in this project the serial part of the parallelized program is 
very small, but the algorithm requires a lot of sending between processes. Since the kongull cluster is used one would expect 
that the sending cost will increase drastically when the program is runned on two or more nodes. It is therefore expected that 
the speedup between 12 and 14 processes will be very small since that is where kongull needs to use 2 nodes instead of just 1.

Another interesting and similar variable to study is the efficiency $\eta = \frac{S_p}{P}$, which has a theoretical upper bound 1.
The trend of this variable should be the same as the one observed in the speedup. 

SOME INFO ON THE LINEAR NETWORK MODEL! 

For larger problem sizes the overhead costs of creating and closing threads, sending procedures and general noise will have
a smaller effect on the total time. One can therefore expect that both the speedup and the efficiency will show a slightly better
trend for large problem sizes. 

OBSERVED RESULTS HERE ??

speedup calculations, theoretic and observed


